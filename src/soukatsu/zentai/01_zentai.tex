\subsection*{2023年度秋学期活動総括}

\writtenBy{\president}{羽田}{秀平}
%\writtenBy{\subPresident}{羽田}{秀平}
%\writtenBy{\firstGrade}{羽田}{秀平}
%\writtenBy{\secondGrade}{羽田}{秀平}
%\writtenBy{\thirdGrade}{羽田}{秀平}
%\writtenBy{\fourthGrade}{羽田}{秀平}

本会の目的である「情報科学の研究,及びその成果の発表を活動の基本に会員相互の親睦を図り,
学術文化の創造と発展に寄与する」ことを達成するため,方針として以下の五つを立てた.
これらについてそれぞれ評価を行うことで2023年度秋学期の総括とする.

\begin{itemize}
    \item 親睦を深める
    \item 規律ある行動
    \item 自己発信力の向上
    \item 会員間の技術向上
    \item 外部への情報発信
\end{itemize}

\subsubsection*{親睦を深める}
    2023年度秋学期活動では,主にプロジェクト活動を通して会員間の親睦を図った.

    学園祭では\secondGrade{}が中心となり,運営及び作品の展示を行うことができたため\secondGrade{}の親睦が深まった.

    一方で一部の\firstGrade{}はプロジェクト活動に参加することができなかったため達成できなかった.

\subsubsection*{規律ある行動}
    2023年度秋学期の方針として,遅刻・欠席連絡と備品整理,
    サークルルームの使用方法の三つの項目からなる行動規範を定めた.

    遅刻・欠席連絡について,活動期間が後になるにつれて連絡が減少してしまった.

    サークルルームの使用方法については,ゴミの処理を怠っていた会員が一部発生していた.

    備品整理については,借主の連絡が無かったため,借りたいのに借りれない状態があった.

    学生部によるサークルルーム点検においても,特に戒告を受けることはなかった.

\subsubsection*{自己発信力の向上}
    自己発信力を向上させるための機会として,2022年度秋学期活動では,
    LTを実施した.一方でAdvent Calendarや会誌の制作を行うことができなかった.

    定例会議におけるLTは,参加率が低く,遅延も多かったため例年より活動が少なかった.

    Advent Calendarや会誌の制作は会員数や本会の状態を鑑みた上で,実施しなかった.
    今後は本会の状態を鑑みた上で,実施するかどうかを決定する.

    これらの理由から自己発信力の向上は達成できなかった.

\subsubsection*{会員間の技術向上}
    会全体の技術力向上について,\secondGrade{}が中心となって作品の展示を行うことができたため達成できた.

    夏季休暇期間及び春季夏季休暇期間では,ハッカソンに参加する会員がいたため技術力向上につながった.

\subsubsection*{外部への情報の発信}
    会外へ活動を発信する機会として,主に本会Webサイトと会公式Twitterが挙げられる.

    本会Webサイトでは,Advent Calendarや会誌の制作を行うことができなかったため記事を投稿することができなかった.

    会公式Twitterでは,OBOG会や学園祭の様子について投稿することができた.
    