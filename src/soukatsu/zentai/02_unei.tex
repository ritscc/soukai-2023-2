\subsection*{運営総括}

%\writtenBy{\president}{大野}{直哉}
\writtenBy{\subPresident}{大野}{直哉}
%\writtenBy{\firstGrade}{大野}{直哉}
%\writtenBy{\secondGrade}{大野}{直哉}
%\writtenBy{\thirdGrade}{大野}{直哉}
%\writtenBy{\fourthGrade}{大野}{直哉}

\subsubsection*{運営サポート}
秋学期は\secondGrade{}中心で運営を行うが,\thirdGrade{}もサポートするという方針を立てたが,
\secondGrade{},\thirdGrade{}が中心となって運営を行っていた.
また必要に応じて\fourthGrade{}に質問し,運営を行った.

\subsubsection*{定例会議}
定例会議は毎週開催するという方針を立てたが,毎週の火曜日に開催できていた.
開催日時については日時調節アプリケーションのtappyを用いて,会員全体にアンケートを取り決定した.
また資料の共有や行事の告知については定例会議と併用してDiscrodで行った.

\subsubsection*{上回生会議}
上回生会議は毎週開催するという方針を立てた.Discord上で情報の共有や議題の周知がされていたため,秋学期については上回生会議は行われなかった.

\subsubsection*{局会議}
原則として局会議は毎週開催するという方針を立てたが,局会議の開催は特に行われなかった.
局会議の代わりにSlackやDiscordを用いて情報の共有を行われていたため,会の運用には問題はなかった.

\subsubsection*{企画}
企画の担当者は常に2人以上とするという方針を立てたが,会員の人数の都合から立命の家以外の企画の担当者は1人であった.
各企画の終了後は,KRPを行うことが出来ていた.
