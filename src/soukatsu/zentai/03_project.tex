\subsection*{プロジェクト活動総括}

%\writtenBy{\president}{尾﨑}{真央}
%\writtenBy{\subPresident}{尾﨑}{真央}
%\writtenBy{\firstGrade}{尾﨑}{真央}
\writtenBy{\secondGrade}{尾﨑}{真央}
%\writtenBy{\thirdGrade}{尾﨑}{真央}
%\writtenBy{\fourthGrade}{尾﨑}{真央}

## 目的
プロジェクト活動の目的は,情報科学の研究をし,その成果の発表を活動の基本として会員相互の親睦を図るとともに学術文化の創造と発展に寄与することとする.

## 目標
個人のみならずグループ活動としての経験を得る
活動を通して技術力の向上を図る
活動によって得られた成果を本会 Web サイトを通して公開する

### プロジェクトの内容
プロジェクト内容は基本的に学習または,研究要素を含むものとする.

### プロジェクトの設立
通年はそのまま
新規プロジェクトの設立を認めます
後期入部者への対応
プロジェクトは以下の条件を満たした場合に設立できるものとする.
リーダーの作成した企画書が,上回生会議で承認されること
メンバーが,募集終了時点でリーダーを含め 3 人以上であること
リーダーが,入会して半年以上経過していること
1 人の会員が,複数のプロジェクトのリーダを担当していないこと
1 人が一度に複数企画書を提出していないこと
上回生会議には企画書を通していない
上回生会議を積極的にする

### メンバー募集
定例会議でリーダーがそれぞれプロジェクトの説明を行い,その次の定例会議までに四回生を除く全会員が一つ以上のプロジェクトに所属する.
ほとんど達成


## プロジェクト発表会
プロジェクトで得た知見や技術を共有する場として成果発表を行う.
発表形式は学祭
