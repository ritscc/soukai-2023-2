\subsection*{\thirdGrade{}総括}

%\writtenBy{\firstGrade}{羽田}{秀平}
%\writtenBy{\secondGrade}{羽田}{秀平}
\writtenBy{\thirdGrade}{羽田}{秀平}
%\writtenBy{\fourthGrade}{羽田}{秀平}

2023年度秋学期の\thirdGrade{}方針は以下の4点であった.

\begin{itemize}
    \item 引き継ぎ文書を残す
    \item サークルルームの利用率を高く保つ
    \item 行事の運営のサポートをする
    \item 新入生の活動への継続的な参加を促す
\end{itemize}

\subsection*{引き継ぎ文書を残す}
引き継ぎ文書を作成し,下回生が今後の活動を円滑に行うことができるようにする.

\subsection*{サークルルームの利用率を高く保つ}
上回生がサークルルームを利用している所を見せることで,利用しやすい雰囲気を作ることができた.
また新入生がロジェクト活動でサークルルームを活用していたので達成することができた.

\subsection*{行事の運営のサポートをする}
\secondGrade{}への引き継ぎを行うとともに今後も本会がイベントを行えるよう\secondGrade{}をサポートする.
特に学園祭では\secondGrade{}が中心となって運営及び作品の展示を行なったので達成することができた.

\subsection*{新入生の活動への継続的な参加を促す}
一部の新入生がプロジェクト活動に参加することができなかったため,達成することができなかった.
新入生に対してアンケートを取ることでやりたいことを把握し,要望があれば機器の購入を行うことが必要だった.
またやりたいことが見つかっていない新入生に対しては,現在の活動を紹介することや,PCに保存されている過去のLT資料を見てもらうことで,本会ではどのような活動が行えるのかを把握してもらう必要があった.
