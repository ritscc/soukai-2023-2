\subsection*{\secondGrade{}総括}

%\writtenBy{\firstGrade}{梶原}{悠人}
%\writtenBy{\secondGrade}{梶原}{悠人}
%\writtenBy{\thirdGrade}{梶原}{悠人}
%\writtenBy{\fourthGrade}{梶原}{悠人}

下記の四点に対しての総括を示す.
\begin{itemize}
    \item 技術力の向上
    \item 引継ぎを行う
    \item 外部イベントの積極的な参加
    \item 全体会議に参加する
    \item 部室が使えた場合,部室に通う
    \item イベント運営
\end{itemize}

\subsection*{技術力の向上}
プロジェクト活動への参加が春学期に比べると良好であり,学園祭でも作品を出していたため達成された.
\subsection*{引継ぎを行う}
一部部員は局配属を行えておらず,仕事の引継ぎも行えていないところがある.
\subsection*{外部イベントの積極的な参加}
一部部員が参加していた.
\subsection*{全体会議に参加する}
春学期に比べると出席率は上がっていたが,より改善する必要がある.
\subsection*{部室が使えた場合,部室に通う}
部室に通っている人は多かったので,達成された.
\subsection*{イベント運営}
学園祭の運営を行うことができていたので,達成された.