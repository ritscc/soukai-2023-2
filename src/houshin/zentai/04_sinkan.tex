\subsection*{新歓方針}

\writtenBy{\president}{羽田}{秀平}
%\writtenBy{\subPresident}{羽田}{秀平}
%\writtenBy{\firstGrade}{羽田}{秀平}
%\writtenBy{\secondGrade}{羽田}{秀平}
%\writtenBy{\thirdGrade}{羽田}{秀平}
%\writtenBy{\fourthGrade}{羽田}{秀平}

\subsubsection*{目的}
新入生歓迎活動を行う目的として以下の二つを挙げる.
\begin{itemize}
\item 新入生に会の活動内容について知ってもらう
\item 新入生に会に興味を持ってもらう
\end{itemize}
これは本会の活動を会外に周知してもらうことで,新入生を歓迎するためのものである.
また,会内及び会外での活動を維持していく上で重要なものである.

\subsubsection*{目標}
目標に関しては,以下の四つを挙げる
\begin{itemize}
\item 企画に参加してもらう
\item 気軽にサークルルームに来てもらう
\item 本会でやりたいことを見つけてもらう
\item 新入生の中長期的な定着
\end{itemize}
目標達成のために,新入生を歓迎することができる期間にサークルルームを見学できることを新入生に周知する.
SNSやホームページで地図や動画を作成するのみならず,実際に案内することでサークルルームの場所を詳しく説明する.
サークルルームではクライアントPCやサーバ,本棚の紹介をすることで会に興味を持ってもらう.
アポイントメントなしで新入生がこれるようにサークルルームを整理し,常に\secondGrade{}の会員が少なくとも二人程度いるようにする.

\subsubsection*{手法}
対面活動ができる場合,以下の手法で新入生を歓迎する.
大学側が主催するイベントに積極的に参加する.
そのために定期的にメールを確認し,申請期限を守る.
参加したイベントでは会誌を配布や,成果物の発表などを行う.
イベントの開催はSNSやホームページで告知する.
本会の定例会議に参加してもらい,会の雰囲気を知ってもらう.
また,ブースやサークルルームの入り口を華やかにし,どういったサークルか分かりやすくする.
サークルルームでの新入生対応では,本会に関することだけでなく,大学生活全体に関する話をする.
