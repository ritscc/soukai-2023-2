\subsection*{Welcomeゼミ方針}

%\writtenBy{\president}{尾﨑}{真央}
%\writtenBy{\subPresident}{尾﨑}{真央}
%\writtenBy{\firstGrade}{尾﨑}{真央}
\writtenBy{\secondGrade}{尾﨑}{真央}
%\writtenBy{\thirdGrade}{尾﨑}{真央}
%\writtenBy{\fourthGrade}{尾﨑}{真央}

##目的
新入生に会に興味を持ってもらう 
新入生に本会の活動を体験してもらう
(入部自体は,4月の第三週目~OKとしていた)
この点に関して規約は無い

##目標
新一回生にとってRCCが居心地のいい空間にする
discordにも入りやすく
活動に積極的に参加してもらう(例会など)
新入生の中長期的な定着


##手法
    どういう形式にするか
        例年と同じく、新入生の希望分野に適した上回生をあてがい開発を行う
        1人の上回生がオーバーワークにならないように調整する
        (*)一回生の進捗管理を共有する(上回生全体で)
        レクチャーの強化((*)の点においても.
        知ってる内容でも,やや砕いて優しい内容で教えてあげること.
        魔材を飲まないように伝える
        宗教的勧誘は厳禁(エディタ、言語等)
            例
            JavaScriptなんて暗黒で恐ろしい言語はダメだ
            純粋関数型言語であるHaskellからやった方がいい

    何をするか
        新入生側がやりたいことを募る方法(Welcomeゼミ)
            分野を絞って一回生同士の交流を増やしたい
            上回生から得意分野を聞き出す
        少しでも教えられるなら書いてほしい
            出来ることをある程度明確に書いてもらうことで,レベル差配置なども考慮
        新入生が開発を希望している物を聞く
            アンケートにスキルを書く欄を作る(任意)
        その分野が得意な上回生とコンビを組む
            ⇒分野は分散してた
            welcomeゼミ担当者は,その週の上回生会議に参加し,執行部とともに割り振る.

        各々のペースでティーチ&メイク
            難易度を全体的に易化
            プロダクト制作の強制はするの?
            作れるといいね程度.
                一回生の多くは,こんなものを作りたいっていう気持ちはあると思うので,それにサポートしてあげる
                一回生はざっくりとしか言えない
                忖度を要求される
                制作物にこだわらずに,成果がでればいい.

        最終日に成果発表
            例会の連絡が終わった後に,行う
            場所:フォレストハウス
                半分以上の人が発表したので増やしたい
                場所の取り方はやぎちゃん(or研推局員)に聞いてください
            例年通り必ず行う
            当日の発表順も事前に決めておくこと.
             
        連絡手段
            discord

### 備考
    ジャンルごとの担当を新歓までに決めておきましょう
    GoogleFormで事前に聞いておく必要がある(3月中が望ましい)
        DTM
        デザイン
        イラスト
        動画制作
        Androidアプリ
        iOSアプリ
        VR
        AR
        3DCG
        ゲーム制作
        人工知能
        画像処理
        音声処理
        サーバ
        セキュリティ
        バイナリ・アセンブラ
        Web
        IoT
        電子工作
        競プロ


        
                    
                
        

    

    

