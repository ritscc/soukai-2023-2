\subsection*{春学期活動方針}

\writtenBy{\president}{羽田}{秀平}
%\writtenBy{\subPresident}{羽田}{秀平}
%\writtenBy{\firstGrade}{羽田}{秀平}
%\writtenBy{\secondGrade}{羽田}{秀平}
%\writtenBy{\thirdGrade}{羽田}{秀平}
%\writtenBy{\fourthGrade}{羽田}{秀平}

本会の目的である
「情報科学の研究,及びその成果の発表を活動の基本に会員相互の親睦を図り,学術文化の想像と発展に寄与する」
ことを達成するため,以下の五つを方針とする.

\begin{itemize}
  \item 親睦を深める
  \item 規律ある行動
  \item 自己発信力の向上
  \item 会員間の技術向上
  \item 外部への情報発信
  \item 持続可能な運営
\end{itemize}

\subsubsection*{親睦を深める}
会員間での親睦を深めることは,本会の目的の一つである.
サークル活動は個人で成り立つものではなく,会員が互いに手を取り合い,助け合うことで実現される.
そのため,会員間の親睦を深めることはサークル活動を成り立たさせる上で必要不可欠である.

本会の主な活動であるプロジェクト活動では,各プロジェクト内での密な連携が要求される.
そのため,回生の枠に囚われず,積極的なコミュニケーションを取っていくことを心がける.

加えて,コミュニケーションの機会となるイベントを企画し,会員間の親睦を深める機会を設ける.
また,新歓期には2023年度と同様に新歓交流会などを企画する事で,新入生が上回生と交流しやすい環境作りを行っていく.

以上のように,2024年度も会員間の親睦を深めるための活動を継続する.

\subsubsection*{規律ある行動}
サークルを運営していく上で一定の規律は必要である.
本項では,サークルが適切かつ円滑に運営されるために,会員が最低限行うべき行動方針を二つ示す.

一つ目に,遅刻及び欠席連絡について述べる.
遅刻・欠席は可能な限りしないことが望ましい.
しかし,やむを得ない事情がある場合は,その理由と遅刻であるならば到着予想時刻を明記した上で連絡すべきである.
その際,不明瞭な理由を記載することは避け,
会全体に伝えるべき内容でない場合は,執行委員長やイベントの主催者に直接伝えることを心がける.
また,これらの連絡は,遅刻・欠席することが判明した場合に早急に行うことを心がける.

二つ目に,会内ルールの徹底について述べる.
深夜のサークルルームの滞在や飲食など他の会員の迷惑となる行為は慎まれるべきである.
また,ホワイトボードの使用後の放置や,ゴミの不始末などサークルルームの環境悪化の原因となる行為もあってはならない.
会員は他会員に対する礼儀及びマナーとして,このような行為の無いように心がけることとする.

\subsubsection*{自己発信力の向上}
自らの意見を相手方に対してわかりやすく伝える能力を身につけることを目標とする.
本会では,LTなどの発表や,会誌及び報告書などの文書作成を通して相手に自分の意見を伝える機会が数多くある.
しかし,これらの活動において相手が理解し難い内容となってしまうような事態は回避すべきである.
そこで,行事における発表に関しては,会員間でレビューを行い,プレゼンテーション能力の向上を図る.
また,文書の作成についても会員間で校正を行い,正しい文書作成能力を身につけることを目指す.

上記の活動を通し,聴衆や読み手に簡潔かつ正確に伝える能力の向上を図る.

\subsubsection*{会員間の技術向上}
会員間で知識や知見の共有を行うことで,新たな分野に触れる機会を提供するとともに,会全体の技術力向上を目指す.
知識や知見の共有の機会としては,LTやプロジェクト活動,勉強会が主に挙げられる.
更に,会員間で切磋琢磨する機会としてハッカソンを行う予定である.

\subsubsection*{外部への情報発信}
会外の人に本会の活動を知ってもらうことを目的とする.
会外へ活動を発信する機会として,本会Webサイトや学園祭,会誌などが主に挙げられる.
2023年度も報告書をWebサイト及び会誌に掲載し,プロジェクト活動内容とその成果について発信していく.
更に,会誌に各会員が自由な内容を発信できるコラムページを設け,各会員個人が行っている活動についても発信する機会を設ける.

\subsubsection*{持続可能な運営}
2024年度の活動では,BKCとOICで活動が分断されることが予想されるため,それぞれの場所で活動を継続することが求められる.
これにあたって,定例会議とプロジェクト活動などで利用する教室は,キャンパス間で連絡を取り合い,事前に予約を入れる必要がある.

上記の活動形態の維持が困難な場合は活動場所の見直しを行うこと及び団体の分離などを検討する.

また団体の分離の案として,本会の一部を分離し,新たな団体を情報理工学部プロジェクト団体として設立することが一つの案である.

次に新入生の勧誘である.2024年度春学期は対面での活動が2023年度と同様に行うことが推測される.
よってブース出展などの突発的な勧誘イベントは参加することを目標とする.

最後に情報理工学部の移転に伴うサークルルームの移転についてである.
執筆時点では,BKCのサークルルームについてはこれまでと同様に利用できることが学生オフィスから連絡されている.
OICのサークルルーム(以下Student Activity Lounge)については,学友会からの連絡がないため,今後の情報を待つこととする.

万が一,Student Activity Loungeが確保できない場合は,学生オフィスや情報理工学部事務室など然るべき機関に問い合わせを行い,新たなサークルルームを確保する.

学生オフィスや学友会から発信されるサークルルームの移転についての情報に従って,本会の会員で移転についての議論を行う.
そして移転後の持続可能な運営方法及び,今後の活動について決定する.
