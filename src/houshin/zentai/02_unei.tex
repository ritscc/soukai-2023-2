\subsection*{運営方針}

%\writtenBy{\president}{大野}{直哉}
\writtenBy{\subPresident}{大野}{直哉}
%\writtenBy{\firstGrade}{大野}{直哉}
%\writtenBy{\secondGrade}{大野}{直哉}
%\writtenBy{\thirdGrade}{大野}{直哉}
%\writtenBy{\fourthGrade}{大野}{直哉}

2023年度の運営について,以下の4点から方針を述べる.
\begin{itemize}
    \item 定例会議
    \item 上回生会議
    \item 局
    \item 企画
\end{itemize}

\subsubsection*{定例会議}
2024年度においても,2022年度同様週1回の定例会議を行う.
定例会議では,局や企画からの連絡や会員全体ですべき議決,LTなどを行う.
開催する時間帯は,2023年度の初めにアンケートをとって決定する.
定例会議で的確な情報を共有するため,各局で確実な情報共有を行う.
情報理工学部のOICへの移転に伴い,2024年度の会員の所属学部によって定例会議の開催場所をOICに変更することを検討する.
また対面での開催が困難な場合は,オンライン上での定例会議の開催を行うことも検討する.

\subsubsection*{上回生会議}
2024年度において,週1回の頻度にて上回生会議を行う.
執行部及び企画担当者は全員参加とする.
欠席の場合は必ず代理人を立てるようにする.
議決権のない会員に関しても参加の意志があればその出席を認めることとする.
局長は局会議内で上回生会議での議題を共有し,局員も議題内容を把握できるよう努める.
開催形式については,Discordのボイスチャンネルを利用することを検討している.
必要に応じて,Discordチャンネルの整備も行っていく.

\subsubsection*{局}
局会議は毎週の開催を強制しないが,議題があれば行うようにする.

2023年度に入会した会員の局配属はまだ行えていないため,2024年度の春に急ぎ行う.
2024年度に入会する会員については,秋学期に配属を行う
希望調査はゴールデンウイーク明けの1か月を目安に行い,執行部で面談を行ったあと,上回生会議で配属先を決定する.

\subsubsection*{企画}
各企画の担当者は基本2人設けるが,会員の人数によっては1人にすることも検討する.
企画の運営は\newGradeIfKouki{}\secondGrade{}と\newGradeIfKouki{}\thirdGrade{}が主体となって行う.
企画書提出から企画の進捗は随時上回生会議にて報告する.また企画終了後にはKPTを上回生会議にて担当者を交えて行う.
