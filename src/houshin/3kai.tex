\subsection*{\newGradeIfKouki{}\thirdGrade{}方針}

%\writtenBy{\firstGrade}{大野}{直哉}
%\writtenBy{\secondGrade}{大野}{直哉}
%\writtenBy{\thirdGrade}{大野}{直哉}
%\writtenBy{\fourthGrade}{大野}{直哉}

\writtenBy{\thirdGrade}{梶原}{悠人}

\subsubsection*{目的と目標}
プロジェクト活動の主な目的は,情報科学の研究,およびその成果の発表を活動の基本とし
会員相互の親睦を図るとともに学術文化の創造と発展に寄与することである.

プロジェクト活動の目標は以下の三つとする.

\begin{itemize}
\item 個人のみならずグループ活動としての経験を得る
\item 活動を通して技術力の向上を図る
\item 活動によって得られた成果を本会Webサイトを通して公開する
\end{itemize}

\subsubsection*{プロジェクト活動の内容}
プロジェクト活動の内容は基本的に学習または,研究要素を含むものとする.

\subsubsection*{プロジェクトの期間}
プロジェクトは半期と通年のどちらかとする.
ただし,秋学期の活動回数が春学期に比べて少ないため,
秋学期のみのプロジェクトが乱立しないよう,
通年を基本とする.

\subsubsection*{メンバー募集}
定例会議でリーダーがプロジェクト説明をし,
その次の定例会議までに\fourthGrade{}を除く全会員がいずれかのプロジェクトに所属する.
また,個人が複数のプロジェクトに参加することも認める.
会員がどのプロジェクトに参加しているかはリストで管理する.

\subsubsection*{プロジェクトの設立}
プロジェクトは,以下の条件を満たした場合のみ設立できるものとする.

\begin{itemize}
\item リーダーの作成した企画書が,上回生会議で承認されること
\item メンバーが,募集終了時点でリーダーを含め3人以上であること
\item リーダーが,新入会員ではないこと
\item 1人の会員が,複数のプロジェクトのリーダを担当していないこと
\item 1人が一度に複数企画書を提出していないこと
\end{itemize}

\subsubsection*{週報}
活動後の週報の提出を義務とする.

\subsubsection*{プロジェクトの運営}
週報が出ていないか週報に活動の継続が難しい旨が記述されていた場合,
プロジェクトのリーダーを上回生会議に招集し,プロジェクトの存続を問う.
その際,リーダーが招集に応じないか,またはプロジェクト存続の意思がない場合,
全ての班員を上回生会議に招集し,リーダーを受け継ぐ意志があるものが存在し,
かつ班員としてプロジェクト活動を行う意志のあるものがリーダーを含め3人以上存在する場合に限り,
プロジェクトを存続するものとする.
この時点でプロジェクトの存続が決定しなかった場合,プロジェクトは解散となる.

\subsubsection*{プロジェクトの解散}
プロジェクト配属後に班員が3人未満となるか,リーダーが欠けた場合,
そのプロジェクトは趣意書をもって理由を記述したのちに,上回生会議によって解散される.
プロジェクトが解散し,いずれのプロジェクトにも属さない会員が存在した場合,
研究推進局員が当該会員と話し合い,他のプロジェクトに配属するものとする.